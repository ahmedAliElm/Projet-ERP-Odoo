\documentclass[12pt,a4paper]{article}
\usepackage[utf8]{inputenc}
\usepackage[french]{babel}
\usepackage{geometry}
\usepackage{graphicx}
\usepackage{hyperref}
\usepackage{listings}
\usepackage{xcolor}
\usepackage{amsmath}
\usepackage{fancyhdr}
\usepackage{titlesec}
\usepackage{enumitem}
\usepackage{booktabs}
\usepackage{array}
\usepackage{longtable}
\usepackage{tikz}
\usepackage{eso-pic}
\usepackage{afterpage}

% Configuration de la page
\geometry{margin=2.5cm}

% Cadre autour du contenu de toutes les pages
\AddToShipoutPictureBG{%
\begin{tikzpicture}[remember picture,overlay]
  \draw[line width=2pt,color=black!70] 
    ([xshift=1cm,yshift=-1cm]current page.north west) -- 
    ([xshift=-1cm,yshift=-1cm]current page.north east) -- 
    ([xshift=-1cm,yshift=1cm]current page.south east) -- 
    ([xshift=1cm,yshift=1cm]current page.south west) -- 
    cycle;
\end{tikzpicture}%
}

% Style par défaut
\fancypagestyle{fancy}{%
  \fancyhf{}
  \fancyhead[L]{\leftmark}
  \fancyhead[R]{}
  \fancyfoot[C]{\thepage}
  \renewcommand{\headrulewidth}{0pt}
  \renewcommand{\footrulewidth}{0pt}
}

% Style pour la première page (sans numéro de page en bas)
\fancypagestyle{firstpage}{%
  \fancyhf{}
  \fancyhead[L]{\leftmark}
  \fancyhead[R]{}
  \fancyfoot[C]{}
  \renewcommand{\headrulewidth}{0pt}
  \renewcommand{\footrulewidth}{0pt}
}

% Style pour les autres pages
\fancypagestyle{plain}{%
  \fancyhf{}
  \fancyhead[L]{\leftmark}
  \fancyhead[R]{}
  \fancyfoot[C]{\thepage}
  \renewcommand{\headrulewidth}{0pt}
  \renewcommand{\footrulewidth}{0pt}
}

\pagestyle{fancy}

% Configuration des couleurs pour le code
\definecolor{codegreen}{rgb}{0,0.6,0}
\definecolor{codegray}{rgb}{0.5,0.5,0.5}
\definecolor{codepurple}{rgb}{0.58,0,0.82}
\definecolor{backcolour}{rgb}{0.95,0.95,0.92}

\lstdefinestyle{mystyle}{
    backgroundcolor=\color{backcolour},
    commentstyle=\color{codegreen},
    keywordstyle=\color{magenta},
    numberstyle=\tiny\color{codegray},
    stringstyle=\color{codepurple},
    basicstyle=\ttfamily\footnotesize,
    breakatwhitespace=false,
    breaklines=true,
    captionpos=b,
    keepspaces=true,
    numbers=left,
    numbersep=5pt,
    showspaces=false,
    showstringspaces=false,
    showtabs=false,
    tabsize=2,
    language=Python
}

\lstset{style=mystyle}

% Configuration des titres
\titleformat{\section}
{\Large\bfseries\color{blue!70!black}}
{\thesection}{1em}{}

\titleformat{\subsection}
{\large\bfseries\color{blue!50!black}}
{\thesubsection}{1em}{}

% Informations du document
\title{
    \vspace{-2cm}
    \textbf{\Huge Module Odoo}\\
    \vspace{0.5cm}
    \Large TP – Gestion des Notes Internes\\
    \Large (Dark Theme)
    \vspace{1cm}
}
\author{
    \textbf{Auteur:} Votre Nom\\
    \textbf{Date:} \today\\
    \textbf{Version:} 3.2
}
\date{}

\begin{document}

\maketitle
\thispagestyle{firstpage}

\newpage
\pagestyle{fancy}
\tableofcontents
\newpage

\section{Introduction}

\subsection{Contexte du Projet}

Ce projet consiste en le développement d'un module Odoo complet pour la gestion des notes internes au sein d'une organisation. Le module a été conçu avec une attention particulière portée à l'expérience utilisateur, notamment avec l'implémentation d'un système de thème sombre/clair et d'une interface moderne et intuitive.

\subsection{Objectifs du Projet}

Les objectifs principaux de ce projet sont :

\begin{itemize}
    \item Créer un système de gestion de notes interne complet et fonctionnel
    \item Implémenter un système de thème sombre/clair avec basculement dynamique
    \item Développer une interface utilisateur moderne et responsive
    \item Fournir des fonctionnalités avancées de gestion (priorités, favoris, échéances)
    \item Assurer une bonne expérience utilisateur avec des notifications et des retours visuels
\end{itemize}

\subsection{Technologies Utilisées}

\begin{itemize}
    \item \textbf{Framework:} Odoo 15+
    \item \textbf{Langage de programmation:} Python 3
    \item \textbf{Base de données:} PostgreSQL
    \item \textbf{Interface:} XML (vues), CSS3, JavaScript
    \item \textbf{Architecture:} MVC (Model-View-Controller)
\end{itemize}

\section{Architecture du Module}

\subsection{Structure du Projet}

Le module suit la structure standard d'un module Odoo :

\begin{lstlisting}[language=bash]
tp_gestion_notes/
├── __init__.py
├── __manifest__.py
├── models/
│   ├── __init__.py
│   ├── note_interne.py
│   └── user_preferences.py
├── views/
│   ├── note_views.xml
│   ├── date_filter_wizard.xml
│   └── wizard_view.xml
├── wizards/
│   ├── __init__.py
│   ├── theme_config_wizard.py
│   └── date_filter_wizard.py
├── security/
│   └── ir.model.access.csv
├── static/
│   └── src/
│       ├── css/
│       │   ├── dark_theme.css
│       │   └── components.css
│       └── js/
│           ├── dark_theme.js
│           └── theme_config.js
└── migrations/
    └── 3.2.0/
        └── post-migration.py
\end{lstlisting}

\subsection{Modèle de Données}

\subsubsection{Modèle Principal : tp.note.interne}

Le modèle principal \texttt{tp.note.interne} hérite de \texttt{mail.thread} et \texttt{mail.activity.mixin} pour bénéficier des fonctionnalités de messagerie et d'activités d'Odoo.

\textbf{Champs principaux :}

\begin{itemize}
    \item \texttt{titre} : Titre de la note (requis, indexé)
    \item \texttt{contenu} : Contenu HTML de la note
    \item \texttt{description} : Description courte de la note
    \item \texttt{auteur\_id} : Auteur de la note (Many2one vers res.users)
    \item \texttt{date\_note} : Date de création de la note
    \item \texttt{date\_echeance} : Date d'échéance (nouveau)
    \item \texttt{statut} : Statut de la note (brouillon, publié, archivé)
    \item \texttt{priority} : Priorité (Basse, Normale, Haute, Urgente) (nouveau)
    \item \texttt{is\_favorite} : Indicateur de favori (nouveau)
    \item \texttt{is\_overdue} : Calculé automatiquement si la note est en retard
    \item \texttt{theme\_couleur} : Couleur du thème
    \item \texttt{is\_dark\_mode} : Mode sombre activé
\end{itemize}

\subsubsection{Méthodes Principales}

\begin{lstlisting}[language=Python]
def action_publier(self):
    """Publie la note avec notification améliorée"""
    
def action_archiver(self):
    """Archive la note avec notification"""
    
def action_restaurer(self):
    """Restaure une note archivée"""
    
def action_toggle_favorite(self):
    """Toggle le statut favori d'une note"""
    
def get_statistics(self):
    """Retourne les statistiques des notes pour le dashboard"""
\end{lstlisting}

\section{Fonctionnalités Principales}

\subsection{Gestion des Notes}

\subsubsection{Création et Édition}

Le module permet de créer et éditer des notes avec :
\begin{itemize}
    \item Un éditeur HTML riche pour le contenu
    \item Un système de titre et description
    \item Attribution automatique de l'auteur
    \item Gestion de la date de création
\end{itemize}

\subsubsection{Statuts des Notes}

Le système de statuts permet de gérer le cycle de vie des notes :

\begin{enumerate}
    \item \textbf{Brouillon} : Note en cours de rédaction
    \item \textbf{Publié} : Note finalisée et visible
    \item \textbf{Archivé} : Note archivée mais conservée
\end{enumerate}

\subsection{Système de Priorité}

Un système de priorité à 4 niveaux a été implémenté :

\begin{table}[h]
\centering
\begin{tabular}{|c|l|}
\hline
\textbf{Niveau} & \textbf{Description} \\
\hline
0 & Basse \\
1 & Normale (par défaut) \\
2 & Haute (⚡) \\
3 & Urgente (🔥) \\
\hline
\end{tabular}
\caption{Système de priorité}
\end{table}

\subsection{Système de Favoris}

Les utilisateurs peuvent marquer des notes comme favorites pour un accès rapide :
\begin{itemize}
    \item Bouton étoile dans toutes les vues
    \item Filtre dédié "Favoris"
    \item Indicateur visuel dans les cartes Kanban
    \item Tri automatique des favoris en premier
\end{itemize}

\subsection{Gestion des Échéances}

\begin{itemize}
    \item Champ \texttt{date\_echeance} pour définir une date limite
    \item Calcul automatique des notes en retard (\texttt{is\_overdue})
    \item Filtre "En retard" dans la recherche
    \item Indicateur visuel avec animation pour les notes en retard
\end{itemize}

\subsection{Filtrage et Recherche}

Le module offre plusieurs options de filtrage :

\begin{itemize}
    \item \textbf{Mes Notes} : Notes de l'utilisateur connecté
    \item \textbf{Favoris} : Notes marquées comme favorites
    \item \textbf{Par statut} : Brouillons, Publiés, Archivés
    \item \textbf{Urgentes} : Notes avec priorité urgente
    \item \textbf{En retard} : Notes avec échéance dépassée
    \item \textbf{Par date} : Filtrage par date spécifique via wizard
\end{itemize}

\subsection{Wizard de Filtrage par Date}

Un wizard dédié permet de filtrer les notes par date :
\begin{itemize}
    \item Sélection de date via un calendrier
    \item Affichage de toutes les notes de la date sélectionnée
    \item Vue filtrée avec titre dynamique
\end{itemize}

\section{Interface Utilisateur}

\subsection{Vues Disponibles}

\subsubsection{Vue Liste (Tree View)}

La vue liste offre :
\begin{itemize}
    \item Édition inline des notes
    \item Colonnes : Favori, Priorité, Titre, Auteur, Date, Échéance, Statut
    \item Actions rapides : Publier, Archiver, Restaurer, Supprimer
    \item Décoration visuelle selon le statut
    \item Tri par défaut : Date décroissante
\end{itemize}

\subsubsection{Vue Formulaire (Form View)}

La vue formulaire comprend :
\begin{itemize}
    \item En-tête avec actions principales
    \item Barre de statut interactive
    \item Section titre avec style amélioré
    \item Zone de description
    \item Métadonnées (Priorité, Auteur, Dates)
    \item Notebook avec onglets :
    \begin{itemize}
        \item Contenu : Éditeur HTML
        \item Informations : Suivi et historique
    \end{itemize}
\end{itemize}

\subsubsection{Vue Kanban}

La vue Kanban offre :
\begin{itemize}
    \item Cartes stylisées avec gradients
    \item Groupement par statut par défaut
    \item Indicateurs visuels :
    \begin{itemize}
        \item Étoile pour les favoris
        \item Icône feu pour les urgentes
        \item Bordure colorée pour les priorités
    \end{itemize}
    \item Métadonnées : Auteur avec avatar, Date relative
    \item Description tronquée automatiquement
\end{itemize}

\subsection{Système de Thème}

\subsubsection{Thème Sombre}

Un thème sombre complet a été implémenté avec :
\begin{itemize}
    \item Variables CSS pour la cohérence des couleurs
    \item Palette de couleurs optimisée :
    \begin{itemize}
        \item Primaire : \#1a1d21
        \item Secondaire : \#2d3136
        \item Tertiaire : \#3d4249
        \item Accent : \#4a90e2
    \end{itemize}
    \item Adaptation de tous les composants Odoo
    \item Scrollbar personnalisée
\end{itemize}

\subsubsection{Bouton de Basculement}

Un bouton flottant permet de basculer entre les thèmes :
\begin{itemize}
    \item Position : Bas droite de l'écran
    \item Icône dynamique : Lune (mode clair) / Soleil (mode sombre)
    \item Animation au survol
    \item Persistance dans localStorage
\end{itemize}

\subsection{Améliorations Visuelles}

\subsubsection{Design Moderne}

\begin{itemize}
    \item \textbf{Cards} : Bordures arrondies, ombres portées, effets de survol
    \item \textbf{Boutons} : Gradients, animations, icônes
    \item \textbf{Typographie} : Hiérarchie claire, espacements optimisés
    \item \textbf{Couleurs} : Palette cohérente avec indicateurs visuels
\end{itemize}

\subsubsection{Animations et Transitions}

\begin{itemize}
    \item Transitions fluides (0.3s ease)
    \item Effets de survol sur les cartes
    \item Animation pulse pour les notes en retard
    \item Transformations au survol des boutons
\end{itemize}

\subsubsection{État Vide Amélioré}

L'état vide (quand il n'y a pas de notes) affiche :
\begin{itemize}
    \item Grande icône stylisée
    \item Titre accrocheur
    \item Description informative
    \item Boîte d'astuce avec icône
\end{itemize}

\section{Améliorations Techniques}

\subsection{Performance}

\subsubsection{Indexation}

Des index ont été ajoutés sur les champs fréquemment recherchés :
\begin{itemize}
    \item \texttt{titre} : Recherche textuelle
    \item \texttt{auteur\_id} : Filtrage par auteur
    \item \texttt{date\_note} : Tri et filtrage par date
    \item \texttt{statut} : Filtrage par statut
    \item \texttt{priority} : Tri par priorité
    \item \texttt{is\_favorite} : Filtrage des favoris
\end{itemize}

\subsubsection{Optimisation des Requêtes}

\begin{itemize}
    \item Utilisation de \texttt{search\_fetch} pour les grandes listes
    \item Calculs optimisés pour les statistiques
    \item Mise en cache des préférences utilisateur
\end{itemize}

\subsection{Sécurité}

\subsubsection{Droits d'Accès}

Le fichier \texttt{ir.model.access.csv} définit :
\begin{itemize}
    \item Accès complet pour le modèle \texttt{tp.note.interne}
    \item Accès au wizard de filtrage par date
    \item Permissions : Lecture, Écriture, Création, Suppression
\end{itemize}

\subsubsection{Validation des Données}

Des contraintes ont été ajoutées :
\begin{itemize}
    \item Validation de la date de note
    \item Validation de la date d'échéance
    \item Vérification de cohérence entre les dates
\end{itemize}

\subsection{Migrations}

Un script de migration a été créé pour :
\begin{itemize}
    \item Ajouter les nouvelles colonnes (\texttt{priority}, \texttt{is\_favorite}, \texttt{date\_echeance})
    \item Créer les index nécessaires
    \item Préserver les données existantes
    \item Appliquer les valeurs par défaut
\end{itemize}

\section{Notifications et Retours Utilisateur}

\subsection{Système de Notifications}

Toutes les actions importantes génèrent des notifications :
\begin{itemize}
    \item \textbf{Publication} : "X note(s) publiée(s) avec succès"
    \item \textbf{Archivage} : "X note(s) archivée(s) avec succès"
    \item \textbf{Restauration} : "X note(s) restaurée(s) avec succès"
    \item \textbf{Suppression} : "X note(s) supprimée(s) avec succès"
    \item \textbf{Favoris} : "Note ajoutée aux/retirée des favoris"
\end{itemize}

\subsection{Messages de Suivi}

Le système de messagerie Odoo est utilisé pour :
\begin{itemize}
    \item Enregistrer les changements de statut
    \item Historique des actions
    \item Traçabilité complète
\end{itemize}

\section{Code Source}

\subsection{Exemple de Modèle}

\begin{lstlisting}[language=Python, caption=Modèle tp.note.interne (extrait)]
from odoo import models, fields, api

class TpNoteInterne(models.Model):
    _name = "tp.note.interne"
    _description = "Note Interne"
    _inherit = ['mail.thread', 'mail.activity.mixin']
    _order = 'date_note desc, create_date desc'
    
    titre = fields.Char(string="Titre", required=True, 
                       tracking=True, index=True)
    contenu = fields.Html(string="Contenu")
    auteur_id = fields.Many2one("res.users", 
                                default=lambda self: self.env.user,
                                tracking=True, index=True)
    date_note = fields.Date(string="Date", 
                           default=fields.Date.today,
                           tracking=True, index=True)
    statut = fields.Selection([
        ("brouillon", "Brouillon"),
        ("publie", "Publié"),
        ("archive", "Archivé"),
    ], string="Statut", default="brouillon", 
       tracking=True, index=True)
    
    priority = fields.Selection([
        ('0', 'Basse'),
        ('1', 'Normale'),
        ('2', 'Haute'),
        ('3', 'Urgente'),
    ], string="Priorité", default='1', 
       tracking=True, index=True)
    
    is_favorite = fields.Boolean(string="Favori", 
                                default=False,
                                tracking=True, index=True)
    date_echeance = fields.Date(string="Date d'échéance",
                                tracking=True)
    
    @api.depends('date_echeance', 'statut')
    def _compute_is_overdue(self):
        today = fields.Date.today()
        for record in self:
            record.is_overdue = (
                record.date_echeance 
                and record.date_echeance < today 
                and record.statut != 'archive'
            )
\end{lstlisting}

\subsection{Exemple de Vue}

\begin{lstlisting}[language=XML, caption=Vue Liste (extrait)]
<tree string="Notes Internes" 
      editable="bottom" 
      class="tp_note_dark tp_list_view"
      decoration-success="statut == 'publie'"
      decoration-warning="statut == 'brouillon'"
      decoration-muted="statut == 'archive'"
      default_order="date_note desc">
    
    <field name="is_favorite" widget="boolean_toggle"/>
    <field name="priority" widget="priority"/>
    <field name="titre" string="Titre"/>
    <field name="auteur_id" widget="many2one_avatar_user"/>
    <field name="date_note" widget="date"/>
    <field name="statut" widget="badge"/>
    
    <button name="action_toggle_favorite" 
            type="object" 
            icon="fa-star"
            invisible="not is_favorite"/>
    <button name="action_publier" 
            type="object" 
            string="Publier" 
            icon="fa-check"
            invisible="statut != 'brouillon'"/>
</tree>
\end{lstlisting}

\section{Conclusion}

\subsection{Résultats Obtenus}

Le module développé répond aux objectifs initiaux :

\begin{itemize}
    \item ✓ Système complet de gestion de notes internes
    \item ✓ Interface moderne et intuitive
    \item ✓ Thème sombre/clair fonctionnel
    \item ✓ Fonctionnalités avancées (priorités, favoris, échéances)
    \item ✓ Bonne expérience utilisateur
    \item ✓ Code maintenable et extensible
\end{itemize}

\subsection{Perspectives d'Amélioration}

Plusieurs améliorations pourraient être apportées :

\begin{itemize}
    \item \textbf{Catégories et Tags} : Système de catégorisation avancé
    \item \textbf{Recherche Full-Text} : Recherche dans le contenu HTML
    \item \textbf{Export PDF} : Génération de PDF pour les notes
    \item \textbf{Partage} : Partage de notes entre utilisateurs
    \item \textbf{Commentaires} : Système de commentaires collaboratif
    \item \textbf{Dashboard} : Tableau de bord avec statistiques visuelles
    \item \textbf{Notifications Email} : Alertes par email pour les échéances
    \item \textbf{API REST} : API pour intégration externe
\end{itemize}

\subsection{Apprentissages}

Ce projet a permis de :

\begin{itemize}
    \item Maîtriser le framework Odoo et son architecture MVC
    \item Développer des interfaces utilisateur modernes avec CSS3
    \item Implémenter des fonctionnalités complexes avec Python
    \item Gérer les migrations de base de données
    \item Optimiser les performances avec l'indexation
    \item Améliorer l'expérience utilisateur avec des animations et notifications
\end{itemize}

\section{Annexes}

\subsection{Structure Complète des Fichiers}

\begin{lstlisting}[language=bash, caption=Arborescence complète]
tp_gestion_notes/
├── __init__.py
├── __manifest__.py
├── models/
│   ├── __init__.py
│   ├── note_interne.py (188 lignes)
│   └── user_preferences.py
├── views/
│   ├── note_views.xml (306 lignes)
│   ├── date_filter_wizard.xml (49 lignes)
│   ├── wizard_view.xml (75 lignes)
│   ├── categorie_views.xml
│   ├── tag_views.xml
│   └── user_preferences.xml
├── wizards/
│   ├── __init__.py
│   ├── theme_config_wizard.py (51 lignes)
│   └── date_filter_wizard.py (31 lignes)
├── security/
│   └── ir.model.access.csv
├── static/
│   ├── description/
│   │   └── icon.png
│   └── src/
│       ├── css/
│       │   ├── dark_theme.css (835 lignes)
│       │   └── components.css (1013 lignes)
│       └── js/
│           ├── dark_theme.js (113 lignes)
│           └── theme_config.js
└── migrations/
    ├── __init__.py
    └── 3.2.0/
        ├── __init__.py
        └── post-migration.py
\end{lstlisting}

\subsection{Statistiques du Projet}

\begin{table}[h]
\centering
\begin{tabular}{|l|r|}
\hline
\textbf{Élément} & \textbf{Nombre} \\
\hline
Fichiers Python & 5 \\
Fichiers XML & 6 \\
Fichiers CSS & 2 \\
Fichiers JavaScript & 2 \\
Lignes de code Python & ~300 \\
Lignes de code XML & ~500 \\
Lignes de code CSS & ~1800 \\
Lignes de code JavaScript & ~150 \\
\hline
\textbf{Total} & \textbf{~2750 lignes} \\
\hline
\end{tabular}
\caption{Statistiques du code source}
\end{table}

\subsection{Champs du Modèle}

\begin{longtable}{|p{3cm}|p{2cm}|p{8cm}|}
\hline
\textbf{Champ} & \textbf{Type} & \textbf{Description} \\
\hline
titre & Char & Titre de la note (requis, indexé) \\
contenu & Html & Contenu HTML de la note \\
description & Text & Description courte \\
auteur\_id & Many2one & Auteur (res.users, indexé) \\
date\_note & Date & Date de création (indexé) \\
date\_echeance & Date & Date d'échéance \\
statut & Selection & Brouillon/Publié/Archivé (indexé) \\
priority & Selection & 0-3: Basse/Normale/Haute/Urgente (indexé) \\
is\_favorite & Boolean & Favori (indexé) \\
is\_overdue & Boolean & En retard (calculé) \\
theme\_couleur & Selection & Couleur du thème \\
is\_dark\_mode & Boolean & Mode sombre activé \\
display\_name & Char & Nom d'affichage (calculé) \\
\hline
\end{longtable}

\subsection{Méthodes du Modèle}

\begin{longtable}{|p{4cm}|p{9cm}|}
\hline
\textbf{Méthode} & \textbf{Description} \\
\hline
\_compute\_display\_name & Calcule le nom d'affichage avec icônes de priorité \\
\_compute\_is\_overdue & Calcule si la note est en retard \\
\_check\_dates & Valide les dates (note et échéance) \\
action\_toggle\_favorite & Bascule le statut favori avec notification \\
action\_publier & Publie la note avec notification \\
action\_archiver & Archive la note avec notification \\
action\_restaurer & Restaure une note archivée \\
action\_supprimer & Supprime la note avec confirmation \\
get\_statistics & Retourne les statistiques pour le dashboard \\
\hline
\end{longtable}

\vspace{2cm}

\begin{center}
\textit{Fin du rapport}
\end{center}

\end{document}

